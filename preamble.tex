\usepackage{amsmath, amsfonts, mathtools, amsthm, amssymb, thmtools}
\renewcommand\qedsymbol{Q.E.D.}
\usepackage{enumitem}
\makeatletter
\usepackage{listings}
\usepackage{xstring}
\usepackage{framed}

% \colorlet{shadecolor}{lightgray!25}

% basics
\usepackage[utf8]{inputenc}
\usepackage[T1]{fontenc}
\usepackage{textcomp}
\usepackage{url}
\usepackage{hyperref}
% \hypersetup{
% 	colorlinks,
% 	linkcolor={black},
% 	citecolor={black},
% 	urlcolor={blue!80!black}
% }
\usepackage{graphicx}
\usepackage{float}
\usepackage{booktabs}
\usepackage{enumitem}
\usepackage{parskip}
\usepackage{emptypage}
\usepackage{subcaption}
\usepackage{multicol}
\usepackage[usenames,dvipsnames]{xcolor}
% \usepackage{cmbright}
\newtheorem*{example}{Example}

\usepackage{mathrsfs}
\usepackage{cancel}
\usepackage{bm}
\newcommand\N{\ensuremath{\mathbb{N}}}
\newcommand\R{\ensuremath{\mathbb{R}}}
\newcommand\Z{\ensuremath{\mathbb{Z}}}
\renewcommand\O{\ensuremath{\emptyset}}
\newcommand\Q{\ensuremath{\mathbb{Q}}}
\newcommand\C{\ensuremath{\mathbb{C}}}
\renewcommand\Re{\ensuremath{\operatorname{Re}}}
\renewcommand\Im{\ensuremath{\operatorname{Im}}}
\DeclareMathOperator{\sgn}{sgn}
\usepackage{systeme}
\let\svlim\lim\def\lim{\svlim\limits}
\let\implies\Rightarrow\space
\let\impliedby\Leftarrow\space
\let\iff\Leftrightarrow\space
\let\epsilon\varepsilon\space
\usepackage{stmaryrd} % for \lightning
\newcommand\contra{\scalebox{1.1}{$\lightning$}}
\let\phi\varphi\space





% correct
\definecolor{correct}{HTML}{009900}
\newcommand\correct[2]{\ensuremath{\:}{\color{red}{#1}}\ensuremath{\to }{\color{correct}{#2}}\ensuremath{\:}}
\newcommand\green[1]{{\color{correct}{#1}}}



% horizontal rule
\newcommand\hr{
	\noindent\rule[0.5ex]{\linewidth}{0.5pt}
}


% hide parts
\newcommand\hide[1]{}



% si unitx
\usepackage{siunitx}
\sisetup{locale = FR}
% \renewcommand\vec[1]{\mathbf{#1}}
\newcommand\mat[1]{\mathbf{#1}}


% tikz
\usepackage{tikz}
\usepackage{tikz-cd}
\usetikzlibrary{intersections, angles, quotes, calc, positioning}
\usetikzlibrary{arrows.meta}
\usepackage{pgfplots}
\pgfplotsset{compat=1.13}


\tikzset{
	force/.style={thick, {Circle[length=2pt]}-stealth, shorten <=-1pt}
}

% theorems
\makeatother
\usepackage{thmtools}
\usepackage[framemethod=TikZ]{mdframed}
\mdfsetup{skipabove=1em,skipbelow=0em}


\theoremstyle{definition}

\declaretheoremstyle[
	headfont=\bfseries\sffamily\color{ForestGreen!70!black}, bodyfont=\normalfont,
	mdframed={
			linewidth=2pt,
			rightline=false, topline=false, bottomline=false,
			% linecolor=ForestGreen, backgroundcolor=ForestGreen!5,
		}
]{thmgreenbox}

\declaretheoremstyle[
	headfont=\bfseries\sffamily\color{NavyBlue!70!black}, bodyfont=\normalfont,
	mdframed={
			thmbox=2pt,
			rightline=false, topline=false, bottomline=false,
			% linecolor=NavyBlue, backgroundcolor=NavyBlue!5,
		}
]{thmbluebox}

\declaretheoremstyle[
	headfont=\bfseries\sffamily\color{NavyBlue!70!black}, bodyfont=\normalfont,
	mdframed={
			linewidth=2pt,
			rightline=false, topline=false, bottomline=false,
			linecolor=NavyBlue
		}
]{thmblueline}


\declaretheoremstyle[
	headfont=\bfseries\sffamily\color{RawSienna!70!black}, bodyfont=\normalfont,
	mdframed={
			linewidth=2pt,
			rightline=false, topline=false, bottomline=false,
			% linecolor=RawSienna, backgroundcolor=RawSienna!5,
		}
]{thmredbox}

\declaretheoremstyle[
	headfont=\bfseries\sffamily\color{RawSienna!70!black}, bodyfont=\normalfont,
	numbered=no,
	mdframed={
			linewidth=2pt,
			rightline=false, topline=false, bottomline=false,
			% linecolor=RawSienna, backgroundcolor=RawSienna!1,
		},
	qed=\qedsymbol\space
]{thmproofbox}

\declaretheoremstyle[
	headfont=\bfseries\sffamily\color{NavyBlue!70!black}, bodyfont=\normalfont,
	numbered=no,
	mdframed={
			linewidth=2pt,
			rightline=false, topline=false, bottomline=false,
			% linecolor=NavyBlue, backgroundcolor=NavyBlue!1,
		},
]{thmexplanationbox}

\declaretheoremstyle[
	headfont=\bfseries\sffamily, bodyfont=\normalfont,
	numbered=yes,
	mdframed={
			linewidth=2pt,
			rightline=false, topline=false, bottomline=false,
		}
]{thmbox}


% \declaretheoremstyle[headfont=\bfseries\sffamily, bodyfont=\normalfont, mdframed={ nobreak } ]{thmgreenbox}
% \declaretheoremstyle[headfont=\bfseries\sffamily, bodyfont=\normalfont, mdframed={ nobreak } ]{thmredbox}
% \declaretheoremstyle[headfont=\bfseries\sffamily, bodyfont=\normalfont]{thmbluebox}
% \declaretheoremstyle[headfont=\bfseries\sffamily, bodyfont=\normalfont]{thmblueline}
% \declaretheoremstyle[headfont=\bfseries\sffamily, bodyfont=\normalfont, numbered=no, mdframed={ rightline=false, topline=false, bottomline=false, }, qed=\qedsymbol ]{thmproofbox}
% \declaretheoremstyle[headfont=\bfseries\sffamily, bodyfont=\normalfont, numbered=no, mdframed={ nobreak, rightline=false, topline=false, bottomline=false } ]{thmexplanationbox}

\declaretheorem[style=thmgreenbox, name=Definition, numbered=yes,numberwithin=chapter]{definition}
\declaretheorem[style=thmgreenbox, name=Definition, numbered=no]{definition*}
\declaretheorem[style=thmbluebox, numbered=no, name=Example]{eg}
% \declaretheorem[style=thmbluebox, numbered=no, name=Example]{example}
\declaretheorem[style=thmbox, name=theorem, numbered=yes,numberwithin=chapter]{theorem}
\declaretheorem[style=thmbox, name=Proposition, numbered=no]{prop}
\declaretheorem[style=thmbox, name=Lemma, numbered=no]{lemma}
\declaretheorem[style=thmbox, name=Lemma, numbered=no]{Lemma}
\declaretheorem[style=thmbox, name=Proposition, numbered=no]{proposition}
\declaretheorem[style=thmbox, name=Proposition, numbered=no]{Proposition}
\declaretheorem[style=thmbox, numbered=no, name=Corollary]{Corollary*}
\declaretheorem[style=thmbox, numbered=no, name=Corollary]{corollary*}
\declaretheorem[style=thmbox, numbered=yes, name=Corollary,sibling=theorem]{Corollary}
\declaretheorem[style=thmbox, numbered=yes, name=Corollary,sibling=theorem]{corollary}
\declaretheorem[style=thmproofbox, name=Proof]{replacementproof}
\renewenvironment{definition}[1][]{
\def\argI{#1}
\ifstrempty{#1}
{
	\begin{definition*}
}
{
	\IfInteger{#1}{
		\setcounter{definition}{#1-1}
		\begin{definition}
		\label{def:\thechapter.#1}
	}
	{
		\begin{definition*}[#1]
			\label{def:#1}
			}
			}
			}
			{
			\IfInteger{\argI}{
			\end{definition}}
			{
		\end{definition*}
	}}

\newenvironment{define}[1][]{
\ifstrempty{#1}
{
	\begin{definition}
}
{
	\setcounter{definition}{#1-1}
	\begin{definition}
		\label{def:\thechapter.#1}}}
		{\end{definition}}

\renewenvironment{proof}[1][]{
% \vspace{-10pt}
\ifstrempty{#1}{
	\begin{replacementproof}
}
{
	\begin{replacementproof}[#1]
		}}{\end{replacementproof}}


\renewenvironment{theorem}[1][]{
\ifstrempty{#1}
{
	\begin{inner}
}
{
	\setcounter{inner}{#1-1}
	\begin{inner}
		\label{thm:\thechapter.#1}}}
		{\end{inner}}
\declaretheorem[style=thmexplanationbox, name=Proof]{tmpexplanation}
\newenvironment{explanation}[1][]{\vspace{-10pt}\begin{tmpexplanation}}{\end{tmpexplanation}}

\declaretheorem[style=thmblueline, numbered=no, name=Remark]{remark}
\declaretheorem[style=thmblueline, numbered=no, name=Note]{note}
% \declaretheorem[style=thmline, numbered=no, name=Remark]{remark}
% \declaretheorem[style=thmline, numbered=no, name=Note]{note}

\newtheorem*{uovt}{UOVT}
\newtheorem*{notation}{Notation}
\newtheorem*{previouslyseen}{As previously seen}
\newtheorem*{problem}{Problem}
\newtheorem*{observe}{Observe}
\newtheorem*{property}{Property}
\newtheorem*{intuition}{Intuition}


\usepackage{etoolbox}
\AtEndEnvironment{vb}{\null\hfill$\diamond$}%
\AtEndEnvironment{intermezzo}{\null\hfill$\diamond$}%
% \AtEndEnvironment{opmerking}{\null\hfill$\diamond$}%

% http://tex.stackexchange.com/questions/22119/how-can-i-change-the-spacing-before-theorems-with-amsthm
\makeatletter
% \def\thm@space@setup{%
%   \thm@preskip=\parskip \thm@postskip=0pt
% }

\newcommand{\oefening}[1]{%
	\def\@oefening{#1}%
	\subsection*{Oefening #1}
}

\newcommand{\suboefening}[1]{%
	\subsubsection*{Oefening \@oefening.#1}
}

\newcommand{\exercise}[1]{%
	\def\@exercise{#1}%
	\subsection*{Exercise #1}
}

\newcommand{\subexercise}[1]{%
	\subsubsection*{Exercise \@exercise.#1}
}



\def\testdateparts#1{\dateparts#1\relax}
\def\dateparts#1 #2 #3 #4 #5\relax{
	\marginpar{\small\textsf{\mbox{#1 #2 #3 #5}}}
}

\def\@lesson{}%
\newcommand{\lesson}[3]{
	\ifthenelse{\isempty{#3}}{%
		\def\@lesson{Lecture #1}%
	}{%
		\def\@lesson{Lecture #1: #3}%
	}%
	\subsection*{\@lesson}
	\testdateparts{#2}
}

% \renewcommand\date[1]{\marginpar{#1}}


% fancy headers
\usepackage{fancyhdr}
\pagestyle{fancy}

\fancyhead[RO,LE]{\@lesson}
\fancyhead[RE,LO]{}
\fancyfoot[LE,RO]{\thepage}
\fancyfoot[C]{\leftmark}

\makeatother

% notes
\usepackage{todonotes}
\usepackage{tcolorbox}

\tcbuselibrary{breakable}
\newenvironment{verbetering}{\begin{tcolorbox}[
			arc=0mm,
			colback=white,
			colframe=green!60!black,
			title=Opmerking,
			fonttitle=\sffamily,
			breakable
		]}{\end{tcolorbox}}

\newenvironment{noot}[1]{\begin{tcolorbox}[
			arc=0mm,
			colback=white,
			colframe=white!60!black,
			title=#1,
			fonttitle=\sffamily,
			breakable
		]}{\end{tcolorbox}}

\newcommand\set[1]{\left\{#1\right\}}

\newcommand{\getCurrentChapterNumber}{%
	\thechapter
}
\declaretheorem[style=thmbox, numbered=yes, name=Theorem, sibling=theorem]{inner}
\declaretheorem[style=thmbox, numbered=no, name=Theorem]{inner*}


\newenvironment{thm}[1][]{
\ifstrempty{#1}
{
	\begin{inner}
}
{
	\setcounter{inner}{#1-1}
	\begin{inner}
		\label{thm:\thechapter.#1}}}
		{\end{inner}}
\newenvironment{thm*}[1][]{
\ifstrempty{#1}
{
	\begin{inner*}
}
{
	\begin{inner*}[#1]
		\label{#1}}
		}
		{\end{inner*}}
\usepackage{import}
\usepackage{xifthen}
\usepackage{pdfpages}
\usepackage{transparent}
\newcommand{\incfig}[1]{%
	\def\svgwidth{\columnwidth}
	\import{./figures/}{#1.pdf_tex}
}

\usepackage{cleveref}

% From 
% https://tex.stackexchange.com/questions/1230/reference-name-of-description-list-item-in-latex
\makeatletter
\def\namedlabel#1#2{\begingroup
	#2%
	\def\@currentlabel{#2}%
	\phantomsection\label{#1}\endgroup
}
\makeatother

\newlist{describe}{description}{1}
\setlist[describe,1]{%
	font=\normalfont\textsf,
	itemindent=0pt,
	wide,
	itemsep=0pt,topsep=2pt,
}

\setlist[enumerate,1]{label={(\alph*)}}
\crefname{page}{page}{page}
\newcommand{\powerset}[1]{\mathcal{P}(#1)}

\newcommand\sol[1]{\par\textit{Solution #1}}
\newenvironment{solution}{\par\begingroup\textit{Solution. }}{\endgroup}
% macro to define a local label
\newcommand\locallabel[1]{\label{\currentprefix:#1}}

% macro to use a local reference
\newcommand\localref[1]{\ref{\currentprefix:#1}}

\newcommand*\eqreflocal[1]{\eqref{\currentprefix:#1}}

% \patchcmd\thmtlo@chaptervspacehack
% {\addtocontents{loe}{\protect\addvspace{10\p@}}}
% {\addtocontents{loe}{\protect\thmlopatch@endchapter\protect\thmlopatch@chapter{\thechapter}}}
% {}{}
% \AtEndDocument{\addtocontents{loe}{\protect\thmlopatch@endchapter}}
% \long\def\thmlopatch@chapter#1#2\thmlopatch@endchapter{%
% 	\setbox\z@=\vbox{#2}%
% 	\ifdim\ht\z@>\z@
% 		\hbox{\bfseries\chaptername\ #1}\nobreak
% 		#2
% 		\addvspace{10\p@}
% 	\fi
% }
% \def\thmlopatch@endchapter{}
