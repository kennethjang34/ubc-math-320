\chapter{Number Systems}

Natural numbers: $\mathbb{N} = \{1, 2, 3, \ldots\}$
\par
Integers: $\mathbb{Z} = \{\ldots, -3, -2, -1, 0, 1, 2, 3, \ldots\}$
\par
Rational numbers: $\mathbb{Q} = \left\{\frac{a}{b} \mid a, b \in \mathbb{Z}, b \neq 0\right\}$\par

\begin{remark}
	Note for real numbers, $\mathbb{Q}$ has holes in it.
	\begin{example}
		$\nexists{p \in  \mathbb{Q}} \text{ s.t } p^2=2$\\
		\begin{proof}
			Assume $\exists{p \in \mathbb{Q}} \text{ s.t } p^2=2$. Then $p = \frac{a}{b}$ where $a, b \in \mathbb{Z}$ and $b \neq 0$. So, $p^2 = \frac{a^2}{b^2} = 2 \implies a^2 = 2b^2$. So, $a^2$ is even $\implies a$ is even. So, $a = 2k$ for some $k \in \mathbb{Z}$. So, $4k^2 = 2b^2 \implies b^2 = 2k^2$. So, $b^2$ is even $\implies b$ is even. So, $b = 2l$ for some $l \in \mathbb{Z}$. So, $a$ and $b$ are both even, which contradicts the fact that $a$ and $b$ are coprime. So, $\nexists{p \in \mathbb{Q}} \text{ s.t } p^2=2$.
		\end{proof}
	\end{example}
\end{remark}


\begin{definition}[Order]
	An order on a set $S$ is a relation $<$ such that:
	\begin{enumerate}
		\item If $a, b \in S$, then exactly one of $a < b$, $a = b$, or $b < a$ is true.
		\item If $a, b, c \in S$ and $a < b$ and $b < c$, then $a < c$.
	\end{enumerate}
\end{definition}


\begin{definition}[Ordered Set]
	An ordered set $S$ is a set with an order $<$.
\end{definition}

\begin{definition}
	Let $S$ be an ordered set.
	A set $E\subset S$ is bounded above if $\exists{\beta \in S} \text{ s.t } \forall{x \in E}: x \leq \beta$.\\
	Similarly, a set $S$ is bounded below if $\exists{\beta \in S} \text{ s.t } \forall{x \in E}: x \geq \beta$.
\end{definition}

\begin{definition}[LUB, GLB]
	Let $S$ be an ordered set and $E \subset S$, $E\neq \emptyset$, with $E$ bounded above. If $\exists \alpha$ s.t. $\alpha$ is an upper bound for $E$ and $\forall \gamma<\alpha$: $\gamma$ is not an upper bound for $E$, then such $\alpha$ is called least upper bound (LUB), or \textit{Supremum}.
	Similarly, if $\exists \alpha$ s.t. $\alpha$ is a lower bound for $E$ and $\forall \gamma>\alpha$: $\gamma$ is not a lower bound for $E$. Then such $\alpha$ is called greatest lower bound (GLB), or \textit{Infimum}.
\end{definition}


\begin{definition}[LUB property]
	An ordered set $S$ has the least upper bound (LUB) property if $\forall E \subset S$ if $E\neq \emptyset$ and $E$ bounded above implies $\exists\sup E \in S$; i.e., Every bounded subset of $S$ has the least upper bound(LUB).
	\begin{example}
		\hfill
		\begin{itemize}
			\item $\mathbb{Z}$ has the LUB property.
			\item $\mathbb{Q}$ does not have the LUB property.
		\end{itemize}
	\end{example}
\end{definition}

\begin{theorem}
	Let $S$ be an ordered set. Then $S$ has the LUB property if and only if $S$ has the $GLB$ property.
	\par
	\hfill
	\begin{proof}
		($\implies$ )Suppose $S$ has the LUB property. Let $B \subset S$ be non-empty and bounded below. Let $L$ be the set of all lower bounds of $B$. Then $L$ is non-empty and bounded above. Let $\alpha = \sup L$. We claim that $\alpha = \inf B$.
		($\impliedby$)Suppose $S$ has the GLB property. Let $E \subset S$ be non-empty and bounded above. Let $U$ be the set of all upper bounds of $E$. Then $U$ is non-empty and bounded below. Let $\beta = \inf U$. We claim that $\beta = \sup E$.
	\end{proof}
\end{theorem}

\begin{definition}[Fields]
	Let $F$ be a set with two operations, addition and multiplication. Then $F$ is a field if the following axioms are satisfied:
	\begin{enumerate}
		\item $a+b=b+a$ and $a \cdot b = b \cdot a$ for all $a, b \in F$ (Commutative laws).
		\item $(a+b)+c = a+(b+c)$ and $(a \cdot b) \cdot c = a \cdot (b \cdot c)$ for all $a, b, c \in F$ (Associative laws).
		\item $a \cdot (b+c) = a \cdot b + a \cdot c$ for all $a, b, c \in F$ (Distributive law).
		\item $\exists 0 \in F$ s.t. $a+0=a$ for all $a \in F$.
		\item $\exists (-a) \in F$ s.t. $a+(-a)=0$ for all $a \in F$.
		\item $\forall x,y \in F: xy \in E$.
		\item $\forall x,y \in F: xy=yx$.
		\item $\exists 1 \in F$ s.t. $a \cdot 1 = a$ for all $a \in F$.
		\item If $a \neq 0$, then $\exists a^{-1} \in F$ s.t. $a \cdot a^{-1} = 1$.
		\item $\forall x,y,z \in F: x(y+z)=xy+xz$
	\end{enumerate}
	\begin{example}\hfill
		\begin{enumerate}
			\item	$\mathbb{Q}$ is a field, while $\mathbb{Z}$ is not a field.
			\item $F_{p}=\{0,1,\ldots, p-1\}$ with mod $p$ arithmetic is a field.
		\end{enumerate}
		Read Text book: 114,115,116,118
	\end{example}
\end{definition}
\begin{definition}[Ordered Field]
	An ordered field $F$ is a field that is an ordered set  such that the following properties are satisfied:
	\begin{enumerate}
		\item If $a,b,c \in F$ and $a<b$, then $a+c<b+c$.
		\item If $a,b \in F$ and $0<a$ and $0<b$, then $0<ab$.
	\end{enumerate}
	\begin{remark}
		We say $x$ is positive if $x>0$ and $x$ is negative if $x<0$.
	\end{remark}

	\begin{example}
		$\mathbb{Q}$ is an ordered field.
	\end{example}
\end{definition}
\begin{theorem}
	$\exists$ an ordered field $\mathbb{R}$ which has the LUB property and contains $\mathbb{Q}$ as a subfield.
\end{theorem}


\begin{theorem}
	\hfill
	\begin{enumerate}[label=(\alph*)]
		\item Arithmetic properties of $\mathbb{R}$:
		      If $x,y \in \mathbb{R}$ and  $x>0$ then $\exists n\in \mathbb{N}$ such that $nx>y$.
		\item $\mathbb{Q}$ is dense in $\mathbb{R}$: If $x,y \in \mathbb{R}$ and $x<y$, then $\exists p \in \mathbb{Q}$ such that $x<p<y$.
		\item $x, y \in \mathbb{R}$ then $\exists \alpha \in \mathbb{R}\setminus \mathbb{Q}$ such that $x<\alpha<y$.
	\end{enumerate}
	\begin{proof}
		\def\currentprefix{arithmetic}
		\begin{enumerate}[label=(\alph*)]
			\item \locallabel{a} Let $A = \{nx \mid n \in \mathbb{N}\}$. Suppose $\forall nx \in A: nx \leq y$. Then $y$ is an upper bound for $A$. So, $A$ has a least upper bound $\alpha$. Since $\alpha - x < \alpha$ as $x>0$, $\alpha -x$  is not an upper bound for $A$. Thus, $\exists m\in \mathbb{N}: mx>\alpha-x$, so $\alpha<(m+1)x\in A$, contradicting the fact that $\alpha$ is a supremum of $A$. Therefore, $\exists n\in \mathbb{N}$ such that $nx>y$.
			\item \locallabel{b}  Since $y-x>0$, by $(a)$, $\exists n\in \mathbb{N}$ such that $n(y-x)>1$. $ny-nx>1$ and therefore, $1+nx<ny$.
			      Let $m\in \mathbb{Z}$ such that $(m-1)\le nx<m$. Such $m$ exists by the extended version of (a).
			      This implies there exists $m \in \N $ such that  $nx<m\le nx+1<ny$. Therefore, $x<\frac{m}{n}<y$.
			\item \locallabel{c} $\exists k\in \mathbb{Q}$ such that $k^2=2$; i.e., $\exists{\sqrt{2} \in \R}$.
			      $0<\sqrt{2}<2$ because if $\sqrt{2}\ge 2$ then $2=\sqrt{2}\cdot \sqrt{2}\ge 2\cdot 2= 4$, which is a contradiction.
			      By \localref{b}, $\exists p\in \Q $ such that $x<p<y$ and $\exists q \in \Q $ such that $x<p<q<y$. Let $\alpha=p+\frac{\sqrt{2}}{2}(q-p)$. Then $x<p<\alpha<q<y$ and $\alpha \not\in \Q $ since otherwise $\sqrt{2}=2\cdot \frac{\alpha-p}{q-p}$ would be rational
		\end{enumerate}
	\end{proof}
	\begin{note}
		(a) can be improved to:
		\begin{equation}
			\forall x,y\in \mathbb{R}, x>0, \exists n \in \mathbb{Z} \text{ such that } (n-1)x\le y<nx.
		\end{equation}
		\begin{proof}
			Case 1: $y\ge 0$. Let $A=\{m\in \mathbb{N}:y<mx\}\subset \mathbb{N}$. By (a), $A\neq \emptyset$. Every non-empty subset of $\mathbb{N}$ has a smallest element. Let $n=\text{smallest element of }A$. Then the inequality holds true.
			Case 2: Let $y<0$, then there exists $n\in\N$ such that $(n-1)x\leq-y<nx$, which implies that (by changing sign for all terms) $-nx<y\leq -(n-1)x$. Hence, the statement holds.
		\end{proof}
	\end{note}
\end{theorem}

\begin{lemma}
	\label{lemma1}
	Let $a,b\in\R$ such that $0<a<b$, then $0<b^n-a^n\leq nb^{n-1}(b-a)$ for some $n\in\N$.
	\begin{proof}
		{
			\begin{align*}
				b^n-a^n & =(b-a)(\underbrace{b^{n-1}+ab^{n-2}+\ldots+a^{n-2}b+a^{n-1}}_{\text{$\footnotesize n$ terms}}) \\
				        & <(b-a)nb^{n-1}
			\end{align*}
		}
	\end{proof}
\end{lemma}

\begin{theorem}
	$\forall x \in \R, x>0, n \in \N: \exists !(\text{unique}) y>0: y^{n}=x$ (we write $y=x^{1/n}=\sqrt{x}^{n}$, the $n^{\text{th}}$ root of $x$).
	\begin{proof}
		Uniqueness: For any $y_1,y_2\in\R$, if $0<y_1<y_2$, then $0<y_1^n<y_2^n$, hence $y_1^n$ and $y_2^n$ cannot both be equal to $x$.\\
		Existence: Let $E_x=\{{t\in\R_{>0}\mid t^n<x}\}$.
		If $E\neq \varnothing$, $E$ is bounded above, hence (by the least-upper-bound property) there exists a $\sup E$. Choose $y=\sup E$. Consider two cases.
		\begin{enumerate}
			\item If $x\leq 1$, then $t_0=\dfrac{x}{2}$ and thereby $t_0^n=\dfrac{x^n}{2^n}<x^n\leq x$ (by assumption that $x\leq 1$).
			\item If $x>1$, then let $t_0=1$, leading to $t_0^n=1<x$.
		\end{enumerate}
		In either case, $t_0\in E$, and hence $E$ is not empty.
		1(a) ($E$ is bounded above)
		Let $\beta=x+1$. Then, $\beta^n=(x+1)^n>x+1>x$. Then, for any $t\in E$, we have that $t^n<x<\beta^n$, hence $t<\beta$, making $t$ an upper bound of $E$.
		\begin{enumerate}
			\item
			      Assuming that $y^n< x$, we find $0<h<1$ such that $(y+h)^n<x$, which leads to $y+h\in E$, something that contradicts with the fact that $y=\sup E$. This is equivalent to finding an $0<h<1$ such that $(y+h)^n-y^n<x-y^n$.
			      By the lemma~\ref{lemma1}, we have $0<(y+h)^n-y^n< n(y+1)^{n-1}h$ for any $0<h<1$. Choose $h$ so that $\frac{(x-y)^{n}}{n(y+1)^{n-1}}$.
			      Then $0<h<1$ still holds and $hn(y+1)^{n-1}<x-y^{n}$, leading to $(y+h)^{n}<x$, and therefore $y+h \in E$. However, this contradicts the fact that $y=\sup E$ as $y+h>y$.
			\item Assuming that $y^n> x$, we find $k>0$ such that $(y-k)^n>x$, which leads to a contradiction since otherwise $y-k$ would be an upper bound for $E$ that's smaller than $y$, which is $\sup{E}$.
			      By the lemma~\ref{lemma1}, $y^n-(y-k)^n\leq ny^{n -1}k<y^n-x$ for any $h<\dfrac{y^n-x}{ny^{n-1}}$. Therefore, $-(y-k)^{n}<-x$, or $x<(y-k)^{n}$. Thus, $y-k$ is also an upper bound of $E$ and $y-k<y=\sup{E}$, which is a contradiction.
		\end{enumerate}
		Since $y^{n}<x$ and $y^{n}>x$ are both contradictions, $y^{n}=x$.
	\end{proof}
\end{theorem}

\begin{definition}[Cut/Dedekind Cut]
	The set $\R$ elements are (Dedekind) cuts, which are sets $\alpha\subset\mathbb{Q}$ such that
	\begin{itemize}
		\item $\forall p\in  \alpha, q \in \Q:q<p \implies q \in \alpha$
		\item No greatest element in $\alpha$
	\end{itemize}
	\begin{example}
		$\alpha=\{p \in \Q \mid p < 0 \} $, $\alpha=\{ {p\in\mathbb{Q}\mid p\leq0\lor p^2<2}\}$
	\end{example}
\end{definition}

\begin{definition}[Order of cuts]
	% \label{def:order_of_cuts}
	For $\alpha,\beta \in \R$, $\alpha<\beta:=\alpha\subset\beta$
\end{definition}
\begin{proof}[test]
	Let $\gamma$ be set of cuts $A$, and show that $\gamma$ is a cut and that $\gamma=\sup A$.
\end{proof}

\begin{theorem}
	There exists an ordered field $\R $ such that $\Q \subset \R$ and $\R$ has the LUB property.
	\begin{proof}
		Let $\R$ be the set of all cuts with:
		\begin{description}
			\item[order] $a<b:=a \subset b$.
			\item[addition] $\alpha+\beta=\{p+q\mid p\in \alpha, q\in \beta\}$.
			\item [multiplication] $\alpha\cdot\beta=\{p\cdot q\mid p\in \alpha, q\in \beta\}$.
		\end{description}
	\end{proof}
\end{theorem}

\subsubsection{Complex Numbers}

\begin{definition}[Complex Field]
	The underlying set is $\mathbb{C}=\{(a,b)| a \in \R , b \in \R \}$\\
	Addition is defined as $(a,b)+(c,d)=(a+c,b+d)$\\
	Multiplication is defined as $(a,b)\cdot(c,d)=(ac-bd,ad+bc)$\\
	Zero element is $(0,0)$\\
	One element is $(1,0)$\\
\end{definition}
\begin{theorem}
	$\mathbb{C}$ is a field.
	\begin{proof}
		Verify the 11 field axioms.
		For just a few axioms:\\
		(M3):\\
		$x=(a,b), y=(c,d), z=(e,f)$.
		$x(yz)=(a,b)(ce-df,cf+de)=(a(ce-df)-b(cf+de),a(cf+de)+b(ce-df))=(ac-bd,ad+bc)(e,f)=(xy)z$\\
		(M4):\\
		$(a,b)(1,0)=(a\cdot 1-b\cdot0,a\cdot 0+b\cdot 1)=(a,b)$\\
		(M5):\\
		$x \neq 0$ means $x=(a,b)$ with $a\neq 0$ or $b\neq 0$. That is, $a^2+b^2>0$. Let $\frac{1}{x}=(\frac{a}{a^2+b^2},\frac{-b}{a^2+b^2})$. Then $x\cdot \frac{1}{x}=(a,b)(\frac{a}{a^2+b^2}+\frac{-b}{a^2+b^2})=(\frac{a^2+b^2}{a^2+b^2},\frac{-ab+ba}{a^2+b^2})=(1,0)$.
	\end{proof}
\end{theorem}
Identification of $\R $ as a subfield of $\mathbb{C}$. Identify $(a,0) \in \mathbb{C}$ with $a \in \R $. Then $(a,0)+(b,0)=(a+b,0)$, $(a,0)(b,0)=(ab,0)$, so we can represent them by $a+b=a+b$, $a \cdot b=a\cdot b$.
Write $i=(0,1)$. $i^2=(0,1)(0,1)=(-1,0)$. So, $i^2=-1$.
$(a,b) \leftrightarrow a+bi$.
Usually write $z=a+bi$ for $z \in \mathbb{C}$.
$\text{Re}(z)=a$,$\text{Im}(z)=b$.
\begin{definition}
	Complex conjugate of $z=a+bi$ is defined as $a-bi$ and denoted by $\overline{z}$
\end{definition}
\begin{note}\hfill
	\begin{enumerate}
		\item $\overline{z+w}=\overline{z}+\overline{w}$
		\item $\overline{zw}=\overline{z}\cdot \overline{w}$
		\item $z+\overline{z}=2\cdot  \text{Re}(z)$
		\item $z-\overline{z}=2i \cdot \text{Im}(z)$
		\item $z\overline{z}=(a+bi)(a-bi)=a^2+b^2\ge 0$, with $=$ if any only if $z=0$
		\item $\frac{1}{z}=\frac{\overline{z}}{z \overline{z}}=\frac{a-bi}{a^2+b^2}$
	\end{enumerate}
\end{note}
\begin{definition}
	$|z|=\sqrt{z\overline{z}}=\sqrt{a^2+b^2}$\\
	In particular, if $z=a \in \R $ then $|z|=\sqrt{a^2}=|a|=
		\begin{cases}
			a  & \text{if } a\ge 0 \\
			-a & \text{if } a<0
		\end{cases}$
\end{definition}

\begin{theorem}
	For $z,w \in \mathbb{C}$,
	\begin{enumerate}
		\item $|z|\ge 0$ with $=$ iff $z=0$
		\item $|z|=|\overline{z}|$
		\item $|zw|=|z|\cdot |w|$
		\item $|\text{Re}(z)\le |z|, |\text{Im}(z)|\le |z|$
		      \begin{proof}
			      Let $z=a+bi$. Then $|\text{Re}(z)|=|a|\le \sqrt{a^2+b^2}=|z|$
		      \end{proof}
		\item $|z+w|\le |z|+|w|$ (Triangle inequality)
		      \begin{proof}
			      \begin{align*}
				      |z+w|^2 & =(z+w)(\overline{z+w})                                   \\
				              & =z\overline{z}+z\overline{w}+w\overline{z}+w\overline{w} \\
				              & =|z|^2+z\overline{w}+w\overline{z}+|w|^2                 \\
				              & = |z|^2+2\text{Re}(z\overline{w})+|w|^2                  \\
				              & \le (|z|+|w|)^2
			      \end{align*}
		      \end{proof}
	\end{enumerate}
\end{theorem}

\begin{thm*}[Cauchy-Schwarz inequality]
	If $a_1,a_n,b_1,b_n \in \mathbb{C}$, then
	\[
		|\sum_{j=1}^{n}{a_{j}\overline{b_{j}}}| \le (\sum_{j=1}^{n}{|a_{j}|^2})^{\frac{1}{2}}(\sum_{j=1}^{n}{|b_{j}|^2})^{\frac{1}{2}}
		.\]
	Interpretation: $(\vec{a},\vec{b})=\sum_{j=1}^{n}{a_j \overline{b_j}}$ defined
	on inner product on $\mathbb{C}^{n}$ and $|(\vec{a},\vec{b})|\le \sqrt{(\vec{a}\cdot \vec{a})(\vec{b} \cdot \vec{b})}$. (Note that $\vec{a}\cdot \vec{b}=\overline{\vec{b} \cdot \vec{a}}$)
	\begin{proof}
		Let $A=\sum{|a_j|^2}$, $B=\sum{|b_j|^2}$, $C=\sum{a_j \overline{b_j}}$
		We can assume 1. $B\neq 0$ because $B=0$ is $0\le 0$, 2. $C\neq 0$ because $C=0$, LHS is 0.\\
		For any $\lambda \in \mathbb{C}$, $0\le \sum_{j=1}^{n}|{a_j+\lambda b_j}|^2=\sum_{j=1}^{n}{(a_j+\lambda b_j)(\overline{a_j}+\overline{\lambda}\overline{b_j})}=\sum_{j=1}^{n}{|a_j|^2}+\lambda \sum_{j=1}^{n}{b_j \overline{a_j}}+\overline{\lambda}\sum_{j=1}^{n}{a_j \cdot \overline{b_j}+|\lambda|^2 \sum_{j=1}^{n}{|b_j|^2}}$.
		Let $\lambda=tC$ for $t \in \R $.\\
		Then $0\le A+\lambda \overline{C}+\overline{\lambda}C+|\lambda|^2B
			=A+2|C|^2t+B|C|^2t^2=p(t)$.
		$p(t)$ is a quadratic function in terms of $t$ and it must be non-negative. Regardless of the value of $t$. Therefore, the discriminant of $p(t)=(2|C|^2)^2-4AB|C|^2=4|C|^2(|C|^2-AB) \le 0$. Since $|C|\ge 0$, $|C|^2\le AB$.
	\end{proof}
\end{thm*}


\begin{definition}[Euclidean $k$-space]
	For $k \in N$, $\R ^{k}:=\{ \vec{x}=(x_1,x_2,\ldots,x_k):x_1,x_2,\ldots,x_k \in \R\}$ with the following properties:
	\begin{describe}
		\item[\namedlabel{def:vector_addition}{Addition}]  $\vec{x}+\vec{y}=(x_1+y_1,x_2+y_2,\ldots,x_k+y_k)$
		\item[\namedlabel{def:scalar_multiplication}{Scalar multiplication}] $\lambda \vec{x}=(\lambda x_1,\lambda x_2,\ldots,\lambda x_k)$
		\item[Inner(dot) product] $(\vec{x},\vec{y})=\sum_{j=1}^{k}{x_j y_j}$, which is bilinear: $(\alpha \vec{x}+\beta \vec{y})\cdot \vec{z}=\alpha \vec{x} \cdot  \vec{z} + \beta \vec{y} \cdot \vec{z}$.
		\item[Norm] $|\vec{x}|=\sqrt{(\vec{x},\vec{x})}={\sum_{j=1}^{k}{|x_j|^2}}^{1/2}$
	\end{describe}
	\begin{remark}
		~\ref*{def:vector_addition} and ~\ref*{def:scalar_multiplication} make $\R ^{k}$ into a vector space.
	\end{remark}
\end{definition}

\begin{theorem}
	Let $\vec{x},\vec{y}, \vec{z} \in \R^{k} $. Then
	\begin{enumerate}[label=(\alph*)]
		\item $|\vec{x}|\ge 0$
		\item $|\vec{x}|=0 \leftrightarrow \vec{x}=\vec{0}$
		\item $|\alpha \vec{x}|=|\alpha| |\vec{x}|$
		\item $|\vec{x} \cdot \vec{y}|\le |\vec{x}||\vec{y}|$ (special case of Cauchy-Schwarz inequality)
		\item $|\vec{x}+\vec{y}|\le |\vec{x}|+|\vec{y}|$ (Triangle inequality)
		      \begin{proof}
			      $|\vec{x}+\vec{y}|^2=(\vec{x}+\vec{y})\cdot (\vec{x}+\vec{y})=
				      |\vec{x}|^2 + 2 \vec{x} \cdot \vec{y}+|\vec{y}|^2
				      \le |\vec{x}|^2 + 2 |\vec{x} \cdot \vec{y}|+|\vec{y}|^2
				      \le (|\vec{x}|+|\vec{y}|)^2
			      $
		      \end{proof}
		\item $|\vec{x}-\vec{y}|\le |\vec{x}-\vec{z}|+|\vec{z} -\vec{y}|$
		      \begin{proof}
			      $|\vec{x}-\vec{y}|=
				      |(\vec{x}-\vec{z})+(\vec{z}-\vec{y})| \le |\vec{x}-\vec{z}|+|\vec{z}-\vec{y}|
			      $
		      \end{proof}
	\end{enumerate}
\end{theorem}

