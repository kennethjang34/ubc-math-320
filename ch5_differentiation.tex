\chapter{Differentiation}
We consider $f:[a,b]\to \R$.

\begin{definition}
	For $f: [a,b]\to \R$ and $x \in [a,b]$, let $f'(x)=\lim_{t\to x}{f(t)-f\frac{x}{t-x}}$ if limit exists.
	Equivalently, $f(t)=f(x)+(t-x)[f'(x)+u(x,t)]$ with $\lim_{t\to x}{u(x,t)}=0$.
\end{definition}

\begin{example}
	\begin{enumerate}[label=(\alph*)]
		\item $f(x)=c \; \text{ for all } x \implies f'(x)=\lim_{t\to x}{\frac{c-c}{t-x}}=0$.
		\item $f(x)=x\; \text{ for all }x\implies f'(x)=\lim_{t\to x}{\frac{t-x}{t-x}}$
		\item $f(x)=e^{x}:=\sum_{n=0}^{\infty}{\frac{x^{n}}{n!}}$.
		      Write $t=x+h$, so $t\to x \Leftrightarrow h\to 0$.
		      $\frac{e^{x+h}-e^{x}}{(x+h)-x}=e^{x}\frac{e^{h}-1}{h}=e^{x}\frac{e^{h}-1}{h}+e^{x}-e^{x}=e^{x}+e^{x}\frac{e^{h}-1-h}{h}$.
		      Let $u(h)=\frac{e^{h}-1-h}{h}$. Then $u(h)=\frac{1}{h}\sum_{n=2}^{\infty}{\frac{h^{n}}{n!}}$, so $|u(h)|=|\frac{1}{h}\sum_{n=2}^{\infty}{\frac{h^{n}}{n!}}|\le |h| \sum_{n=2}^{\infty}{\frac{1}{n!}}=(e-2)|h|$ (note for $n\ge 2, |h^{n-1}|\le |h|$ if $|h|\le 1$). Hence, $u(h)\to \infty$ as $h\to 0$ and therefore $f'(x)=e^{x}$.
		      \begin{remark}
			      $e^{x}:=\sum_{n=0}^{\infty}{\frac{x^{n}}{n!}}$ is well defined.
			      $e^{1}=\sum_{n=0}^{\infty}{\frac{1}{n!}}=e$. Regarding it as a power series, its radius of convergence is $R=\infty$. Also, $e^{x+y}=e^{x}e^{y}$ using definition 3.48 of product series (Rudin's p.178-180).
		      \end{remark}
		      \begin{note}
			      $f'(x)$: Lagrange's notation, $\frac{df}{dx}$: Leibnitz notation
		      \end{note}
	\end{enumerate}
\end{example}

\begin{thm}[2]
	Suppose $f:[a,b]\to \R$ and $f'(x)$ exists. Then $f$ is continuous at $x$.
	\begin{proof}
		The existence of $f'(x) \Leftrightarrow f(t)=f(x)+(t-x)[f'(x)+u(x,t)]$ with $\lim_{t\to x}{u(x,t)}=0$. Let $t\to x$. $\lim_{t\to x}{f(x)+(t-x)[f'(x)+u(x,t)]}=f(x)+0[f'(x)+0]=f(x)$, so $\lim_{t\to x}{f(t)}=f(x)$; i.e., $f$ is continuous at $x$.
	\end{proof}
	\begin{remark}
		The converse is false; e.g., $f(x)=|x|$ is continuous for all $x$, but $f'(0)$ does not exist.
	\end{remark}
\end{thm}

\begin{thm}[3]
	If $f:[a,b]\to \R$ and $g:[a,b]\to \R$ are both differentiable at $x$ then so are $f+g,fg,\frac{f}{g}(\text{ if } g(x)\neq 0)$, and $(f+g)'(x)=f'(x)+g'(x),(fg)'(x)=f'(x)g(x)+f(x)g'(x),(\frac{f}{g})'(x)=\frac{f'(x)g(x)-g'(x)f(x)}{g(x)^2}$. \end{thm}
\begin{proof}[Only the quotient rule]
	$h(t)-h(x)=\frac{f(t)}{g(t)}-\frac{f(x)}{g(x)}=\frac{1}{g(t)g(x)}[(f(t)g(x)-f(x)g(x))-(f(x)g(t)-f(x)g(x))]$.\\
	Then $\frac{h(t)-h(x)}{t-x}=\frac{1}{g(t)g(x)}\left[ \frac{f(t)-f(x)}{t-x}g(x)-f(x)\frac{g(t)-g(x)}{t-x} \right]$.
	Let $t\to x$. $h'(x)=\frac{1}{g(x)^2}\left[ f'(x)g(x)-f(x)g'(x) \right] ]$.
\end{proof}
\begin{remark}
	By induction, $(f_1\cdots f_n)'=f_1'f_2\cdots f_n+f_1 f_2'f_3\cdots f_n+ \cdot f_1 f_2 \cdots f_n'$.
	\begin{example}
		For $n=2,3$,
		$\dfrac{d}{dx}x^{n}=nx^{n-1}$ and we know this already for $n=0,1$.
		For $n=-1,-2$, let $m=-n>0$. Then $\dfrac{d}{dx}x^{n}=\dfrac{d}{dx}\dfrac{1}{x^{m}}=\dfrac{(\dfrac{d}{dx}1)x^{m}-(\dfrac{d}{dx}x^{m})1}{(x^{m})^2}=\dfrac{0x^{m}-mx^{m-1}1}{x^{2m}}=-mx^{m-1}=nx^{n-1}$. Hence, $\forall_{n \in \Z}: \dfrac{d}{dx}^{n}=nx^{n-1}$.
	\end{example}
\end{remark}

\begin{thm}[5][Chain Rule]
	Suppose $f:[a,b]\to \R$, $f'(x)$ exists for some $x \in [a,b], f([a,b]) \subset I$, where $I$ is some interval in $\R$. Suppose $g: I\to \R$ and $g'(f(x))$ exists. Then $g \circ f$ is differentiable at $x$ and $(g\circ f)(x)'=g'(f(x))f'(x)$
	\begin{proof}
		Let $h(t)=(g\circ f)(t)=g(f(t))$ for $t \in [a,b]$. Fix $x \in [a,b]$ where $f'(x)$ exists. We know:
		\begin{enumerate}
			\item $f(t)-f(x)=(t-x)[f'(x)+u(t)]$ with $\lim_{t\to x}{u(t)}=0$.
			\item With $y=f(x)$, $g(s)-g(y)=(s-y)(g'(y)+v(s))$ with $\lim_{s\to y}{v(s)}=0$
		\end{enumerate}
		As $\dfrac{h(t)-h(x)}{t-x}=\dfrac{g(f(t))-g(f(x))}{t-x}$. By (2), $\dfrac{g(f(t))-g(f(x))}{t-x}[g'f(x)+v(f(t))]$.
		Let $t\to x$. Then $\text{ RHS } \to f'(x)[g'(f(x))+0]$ since $f(t)\to f(x)$ by continuity of $f$ at $x$. Therefore, $h'(x)=f'(x)g'(f(x))$.
	\end{proof}
	\begin{note}
		Suppose you produce $f(t)$ meters of wire by time $t$; i.e., rate of wire production is $f'(t) \;m/x$. Also suppose you get $\$\; g(l)$ for $l$ meters of wire; rate of profit is $g'(l) \;\$/m$. Then the rate of earning by time $t$ is $g'(f(t))f'(t) \;\$/m$.
	\end{note}
\end{thm}

\begin{example}
	\begin{enumerate}
		\item $\dfrac{d}{dx}e^{x^2}=2xe^{x^{2}}$
		\item $f(x)=\begin{cases}
				      x \sin{\dfrac{1}{x}} & \text{ if } x\neq 0 \\
				      0                    & \text{ if } x=0
			      \end{cases}$
		      \begin{remark}
			      \begin{enumerate}
				      \item
				            $f$ is continuous on $\R$, including at $x=0$.
				            \begin{proof}
					            $|f(x)|\le |x|$, so by the Squeeze theorem, $\lim_{x\to 0}{f(x)}=0=f(0)$.
				            \end{proof}
				      \item $f$ is differentiable on $x\neq 0$, but not differentiable at $x=0$.
				            For $x\neq 0$, $f'(x)=\sin{\dfrac{1}{x}}+x(\cos{\dfrac{1}{x}})(\dfrac{-1}{x^2})=\sin{\dfrac{1}{x}}-\dfrac{1}{x}\cos{\dfrac{1}{x}}$.
				            For $x=0$, $\dfrac{f(t)-f(0)}{t-0}=\dfrac{t \sin{\dfrac{1}{t}}}{t}=\sin{\dfrac{1}{t}}$, which does not converge. Therefore, $f$ not differentiable at $x=0$.
			      \end{enumerate}
		      \end{remark}
		\item Let $f(x)= \begin{cases}
				      x^2 \sin{\dfrac{1}{x}} & \text{ if } x\neq 0 \\
				      0                      & x=0
			      \end{cases}$.
		      \begin{enumerate}
			      \item
			            $f$ is continuous in $\R$ including at $x=0$ ($\because |f(x)|\le |x^2|$).
			      \item $f$ is differentiable in $\R$ including at $x=0$.
			            \begin{proof}
				            For $x\neq 0$, $f'(x)=2x \sin{\dfrac{1}{x}}+x^2(\cos{\dfrac{1}{x}})(\dfrac{-1}{x^2})=2x \sin{\dfrac{1}{x}}-\cos{\dfrac{1}{x}}$.
				            For $x=0$, $\dfrac{f(t)-f(0)}{t-0}= \dfrac{t^2 \sin{\dfrac{1}{t}}}{t}=t \sin{\dfrac{1}{t}}\to 0$ as $t\to 0$. Hence, $f'(0)=0$.
				            HOWEVER! $f'$ is not continuous at $x=0$, because $\lim_{x\to 0}{f'(x)}$ does not exist.
			            \end{proof}
		      \end{enumerate}
	\end{enumerate}
\end{example}
